\documentclass{article}
\usepackage{amsmath}
\usepackage{array}
\usepackage{booktabs}

\begin{document}

\section{Project Expenditure Profile}

\subsection{Fundamental Expenditure Behavior}

The underlying distribution of expenditures (independent of cost escalation and schedule delay), $I_0(t)$, is assumed to be a sinusoid:

\begin{equation}
  I_0(t,d) = K \frac{\pi}{2d} \, \sin\left( \frac{\pi t}{d} \right),
\end{equation}

where $d$ is the total duration of the project (in years) and $K$ is the initial estimate of the total final project cost.

For $d$ held constant, the cumulative expenditures on the project at time $t$, $E_0(t)$, is the integral of the incremental spend function:

\begin{equation} \label{e0-def}
  \begin{split}
    E_0(t) & = \int_0^t {I_0(t') \, \mathrm{d}t'} \\
      & = - \frac{K}{2} \left[ \cos \left( \frac{\pi t}{d} \right) - 1 \right]
  \end{split}
\end{equation}

These functions have the feature that evaluating $E_0$ at the end of the project (time $t$ equal to $d$, the project duration) reproduces the total project cost:

\begin{equation}
  E_0(d) = K.
\end{equation}


\subsection{Schedule Slip}
For eq. \eqref{e0-def} to be valid, $d$ must be computed ahead of time and not changed during the course of the project.
(On-the-fly schedule slip will be implemented in future versions.) \\

Historically, schedule slip typically occurred continuously throughout all stages of projects, with the effects being most severe towards the end of each project.
DOE/EIA-0485 provides data for cost and schedule escalation for nuclear project in the US up to 1986. 
For each project, expected completion dates are provided at each of six milestones: Project start, 25\% completion, 50\% completion, 75\% completion, 90\% completion, and 100\% completion. \\

Using this data, probability distributions were developed for the schedule slip factors during each phase (0-25\% completion, 25-50\% completion, and so on).
Various probability distributions were examined against the data; the gamma distribution produced the closest match.
Data for each completion stage were fitted to gamma distributions $S_i$ with the following parameters:

%\begin{table}[!htbp]
\begin{center}
$\mathrm{S_i} \sim \Gamma \left( \alpha_i, \beta_i \right)$\\
\begin{tabular} {| c | c | c |} \hline
%  \multicolumn{5}{c}{$\mathrm{X} \sim \Gamma \left( \alpha, \beta \right)$} \\
  Project Stage & $\alpha_i$ & $\beta_i$ \\ \hline
  0-25\% & 2.422 & 0.419 \\ \hline
  25-50\% & 4.219 & 0.166 \\ \hline
  50-75\% & 7.923 & 0.018 \\ \hline
  75-90\% & 3.555 & 0.227 \\ \hline
  90-100\% & 8.758 & 0.026 \\ \hline
\end{tabular}
\end{center}
%\end{table}

Each project has a statically-defined initial schedule estimate, $d_0$, pulled from the reactor project profile.
This duration is divided among five project stages of equal length.
Each stage's duration $d_i$ is then escalated by a factor generated from the appropriate probability distribution.

\begin{equation}
  d_i = \frac{\Gamma \left( \alpha_i, \beta_i \right) \, d_0}{5}
\end{equation}

The total project duration $d$ is the total of the durations of all stages:

\begin{equation}
  d = \sum_{i=1}^5 d_i
\end{equation}

\subsection{Cost Escalation}

For most historical projects, cost escalation occurred in all stages of the project and became more severe over time.
The new incremental and cumulative expenditure functions are named $I_e(t)$ and $E_e(t)$, respectively.
The behavior of these functions is predicated on an escalated total final project cost at time $d$:

\begin{equation}
  E_e(d) = \epsilon \, E_0(d)
\end{equation}

The final total escalation factor, $\epsilon$, is calculated by:

\begin{equation*}
  \epsilon = \Gamma \left( \alpha, \beta \right),
\end{equation*}

where $\Gamma$ is the gamma distribution, $\alpha$ is the shape parameter, and $\beta$ is the rate parameter. At present, values of $\alpha = 1.2$ and $\beta = 0.6$ are being used.\\

This model escalates the fundamental expenditure profile $I_0(t)$ with an exponential escalator term, $e^{\alpha t}$.
An appropriate exponential growth rate $\alpha$ must be determined, in order to reproduce the desired value of $E_e(d)$:

\begin{equation}
  \int_0^d \! {I_0(t) \, e^{\alpha t} \, \mathrm{d}t} = E_0(d) \, \left( 1 + \epsilon \right)
\end{equation}

Evaluating the integral and some algebraic manipulation produce the analytically unsolvable expression:

\begin{equation} \label{non-an}
  e^{\alpha d} - \frac{2 (1 + \epsilon) {\alpha}^2 d^2}{\pi^2} = 2 (1 + \epsilon) - 1
\end{equation}

However, inputting some likely values for the parameters $\alpha$, $\epsilon$, and $d$ demonstrates that the second term is negligible over value ranges of interest.

\begin{center}
\begin{tabular}{| c | c | c | c | c | c |}
  \hline
  $\alpha$ & $\epsilon$ & d & $e^{\alpha d}$ & $\frac{2 (1 + \epsilon) {\alpha}^2 d^2}{\pi^2}$& $2(1 + \epsilon) - 1$  \\ \hline
  0.01 & 0.1 & 10 & 1.001 & 0.00223 & 1.2 \\ \hline
  0.03 & 0.5 & 12 & 1.43 & 0.039 & 2.0 \\ \hline 
\end{tabular}
\end{center}

Thus, $e^{\alpha d} \gg \frac{2 (1 + \epsilon) {\alpha}^2 d^2}{\pi^2}$ and $2(1 + \epsilon) - 1 \gg \frac{2 (1 + \epsilon) {\alpha}^2 d^2}{\pi^2}$, and the second term is neglected. Equation \eqref{non-an} therefore reduces to:

\begin{equation}
  \alpha = \frac{1}{d} \ln \left(1 + 2 \epsilon \right)
\end{equation}

\end{document}
